\documentclass[letterpaper]{article}
\usepackage[margin=1in]{geometry}
\usepackage{hyperref}
\usepackage{indentfirst}
\title{Emergent Technology in Tech Startup \protect\\ Software Engineering Projects}
\author{}
\date{}
\begin{document}
  \maketitle
  \section{Principal Investigator}

    \textbf{Dr. Kevin Buffardi}, Assistant Professor, Computer Science 
    Department. PI's 6-year tenure review is scheduled for 2019-2020 academic 
    year. His expertise includes: Software Engineering, Software Testing, 
    Human-Computer Interaction, and Computer Science Education Research.

  \section{Topic of Research}
  Contemporary methods in Software Engineering involve developers working in close contact with business people and customers to incrementally produce working software and solicit feedback to adapt to refined understanding of the stakeholders' needs. While many schools adopt problem-based learning (PBL) in Software Engineering by assigning students to semester-long projects, traditional approaches leave students with a classroom experience that is unrealistic when compared to Software Engineering practices in industry. Most notably, projects do not typically involve real customers and business clients, who apply external pressures on projects. 

  To address the disconnection between academic and industry experiences, we introduced the Tech Startup model: an approach to teaching Software Engineering in partnership with a business class. The Tech Startup model combines leading methodologies in software engineering and entrepreneurship with compatible, iterative approaches. Our preliminary studies have found improved realism and students' closer adherence to industry techniques. However, we also observed that the vast majority of the project ideas were initiated by entrepreneurs and that software engineers tended to consider themselves as subordinates to the entrepreneurs. To the contrary, the teams should reflect a partnership without explicit hierarchy as both parts of the team work together to create tech startups.

  Moreover, each year in the United States, fifteen to twenty technology companies are founded that eventually surpass \$100 Million USD in revenue\footnote{Kedrosky, P. (2013). "The Constant: Companies that Matter." Available at SSRN 2262948.}. As evidenced by technology leaders Google\footnote{Google: Our History in Depth. https://www.google.com/about/company/history/ Accessed October 2017.} and Facebook\footnote{Phillips, S. (2007). A brief history of Facebook. The Guardian, 25(7).}, successful companies have originated from innovative ideas and software generated by engineering students. However, our previous surveys revealed that when choosing projects for our Software Engineering class, engineering students were motivated more by the technology involved in a project than by its subject or its likelihood of becoming commercially successful. In other words, students are more motivated by innovative, technical challenges than they are by business viability. Encouraging students to explore challenging technologies should also empower them to take greater agency in proposing innovative projects and to assume more leadership and ownership within the project teams.

  Consequently, we propose encouraging Software Engineering students to pursue innovative projects that involve emergent technologies. At the beginning of the semester, students will be prompted with an introduction to the emergent technology fields of Virtual/Augmented Reality (VR/AR) and the Internet of Things (IoT). Accordingly, they will be provided access to lab equipment to enable them to propose and develop projects that leverage these fields. Consequently, we expect Chico State engineers to be highly-motivated to explore leading-edge technology and engage in innovative, interdisciplinary, entrepreneurial projects.\newpage

  \section{Research Plan}
  After the introduction to emergent technologies, students will brainstorm and propose projects and we will collect pre- and post-semester surveys that evaluate students' motivations and experiences over the semester. These surveys will be compared to 3 years of baseline data from previous semesters of Software Engineering team projects. We hypothesize that given the opportunity to work on exciting, emergent technology, engineers will propose more projects associated that use those technologies and that those projects will demonstrate improved business and technical successes, including: intellectual property/patents, incorporation, business funding, and revenue generation. To prepare future classes to take advantage of the emergent technologies, student research assistants will survey the literature on each technology, produce a list of example real-life problems that the technology might address, and write a technical guide to getting started developing with the technology. Assistants will also participate in co-authoring research papers that result from the studies of incorporating emergent technologies in the Tech Startup model.

  While underrepresented in STEM, Hispanic/Latino students are more likely to seek entrepreneurial career paths; however, on average, they also have less funding to support their businesses\footnote{Emily Fetsch, "Left Behind? The New Generation of Latino Entrepreneurs" \url{http://www.kauffman.org/blogs/growthology/2015/04/the-new-generation-of-latino-entrepreneurs/} Ewing Marion Kauffman Foundation, April 13, 2015.}. As a Hispanic-Serving Institution, we have the unique opportunity to explore a potential approach to improving underrepresented recruitment and retention as well as to foster student interest and success in entrepreneurship. Accordingly, we will study demographic differences in motivation and engagement as demonstrated in the project surveys. We will also work with the MESA Engineering Program to solicit assistantships who meet eligibility criteria: (1) Hispanic, or first generation college student, or low income, AND (2) Enrolled as a student in one of the ECC majors. Solicitation will only waive these criteria if either assistantship search fails after 30 days of advertisement.

  \section{Budget}
  With a donation from Alumni, the PI is purchasing an HTC Vive virtual reality headset with no cost to the project. However, the project requires lab equipment including: X Glass Enterprise (formerly Google Glass) for AR applications,  and Amazon Echo and Raspberry Pi kits for building IoT devices and services. \$2000 of the budget will be allocated for lab equipment. Two student assistants will be hired for the responsibilities described above for one-semester scholarships of \$1500 each, totaling \$3000. A total budget of \$5000 is requested.

  \section{Collaborators}
  I am collaborating with Dr. Colleen Robb and David Rahn of the Entrepreneurship program and have coordinated external evaluations of Tech Startup projects with Wendy Porter, director of ChicoStart. Future external funding will be sought from National Science Foundation's Emerging Frontiers and Multidisciplinary Activities (EFMA) division, with a proposal submission planned for February, 2019. Partnerships with businesses and investors may provide additional support.

  \section{Entrepreneurship}
  The project directly involves students in entrepreneurial projects that include interdisciplinary collaboration with the College of Business. As explained in the preceding sections, we expect the tech startup projects to innovate in emergent technological areas of Virtual/Augmented Reality and the Internet of Things. The projects will potentially produce intellectual propoerty and launch business startups. Meanwhile, integrating emergent technologies with the Tech Startup model should increase student motivation while they practice contemporary methodologies and gain experience collaborating with Entrepreneurship students and faculty mentors.

\end{document}