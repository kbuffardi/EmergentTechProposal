\documentclass[letterpaper]{article}
\usepackage[margin=1in]{geometry}
\title{Emergent Technolgoy in Tech Startup Software Engineering Projects}
\author{}
\date{}
\begin{document}
  \maketitle
  \section{Principal Investigator}

    \textbf{Dr. Kevin Buffardi}, Assistant Professor, Computer Science 
    Department. PI's 6-year tenure review is scheduled for 2019-2020 academic 
    year. His expertise includes: Software Engineering, Software Testing, 
    Human-Computer Interaction, and Computer Science Education Research.

  \section{Topic of Research}
  A description of the innovative idea you plan to pursue. Categories for
  investigation include technical development (Boyer scholarship of discovery) 
  or teaching innovation (Boyer scholarship of teaching). What is your basic 
  hypothesis or question being addressed and expected outcomes? 40\% weight

  \section{Research Plan}
  A basic description of the research plan – experimental design/approach, data
  collection and assessment, description of how student assistants be utilized to conduct the work and how they will be included in any scholarship that results from this or future related work. 20\% weight.

  \subsection{Inclusion}
  To promote equity and encourage diversity in Computer Science, the PI will work with the MESA Engineering Program (MEP, via Director Paul Villegas) initially to solicit assistantship applications. According to their eligibility requirements, students must meet criteria of:
  \begin{itemize}
    \item Hispanic, or first generation college student, or low income, AND
    \item Enrolled as a student in one of the ECC majors
  \end{itemize}
  Call for applications will only expand to include students who do not meet these criteria if the initial solicitation fails to fulfill an assistantship position after 30 days of advertisement.

  \section{Budget}
  A budget aligned with the project description and research plan that outlines the use of the \$5,000, primarily focused on student assistant support (required) and materials and supplies. 10\% weight.

  \section{Collaborators}
  Who are you currently working with on your research idea (industry, government agency, CSUC collaborators, etc.), what is their involvement in the research? What are the external funding agencies (examples – Cal State agencies, DOE, NSF, EPA; industries also qualify) and the specific program within the agency/indusstry you’ve targeted for future external funding opportunities, and the cycle (dates) for the next RFP opportunity? 10\% weight.

  \section{Entrepreneurship}
  What are the entrepreneurial aspects of your work and potential outcomes (potential for developing intellectual property, building/strengthening relationships with specific industries, dissemination of the technical and broad results/outcomes via courses you teach through curriculum development or integration? 20\% weight.




\end{document}